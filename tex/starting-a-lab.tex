\chapter{Hey, Where is the Stuff?}
\textit{An introduction to creating your school’s first laboratory}
In countries that are or have been rapidly expanding their education system, a PCV may arrive at her school to find that a laboratory is merely a twinkle in the headmaster’s eye, a distant item on a long list of planned projects.  There may not be a space set aside for doing science experiments, a single beaker for holding colorful liquids, or even a fellow science teacher trained to perform experiments.  For the PCV who wishes to put science in the curious hands of her students and develop an effective laboratory that the school can operate for years to come, we recommend the following guide.
\begin{itemize}
\item{Knowing Thy School}
\item{Knowing Thy Syllabus}
\item{Knowing Thy Self}
\item{Getting the Goods}
\item{Setting up Shop}
\item{Up-Scaling}
\item{Now What?}
\end{itemize}

\section{Knowing Thy School}
Not all schools are created equal, so forget whatever you heard other PCVs say about their schools.  You can certainly learn from their experience, but your school will have its own quirks, strengths, passions, and needs.  The interest in hands-on science may be universal among students, but buy-in from teachers and communities is not necessarily, and you will need to know from the start whether you’re working for the headmaster on this lab project, or whether you’re the only person within a 10-km radius who knows how to electrolyze copper sulfate solution and therefore the brains behind this operation.

As soon as you are comfortable, talk to the staff about the existing infrastructure for doing science at your school.  Is there a dusty cabinet somewhere in the store room that has retort stands and bottles of acid?  Is there an extra classroom designated or available for lab use, and can you secure it as such?  

Here are some things to look for:
\begin{enumerate}
\item{Stuff.  Find out what the school already owns in the way of apparatus and chemicals.}
\item{Know-how.  Find out what your fellow teachers know and can do.  If they were trained to do experiments or have ideas for how to get the lab started, they are your first and best resource.}
\item{Safety.  If you’re going to be working with chemicals and electricity, you’ll need to know what first aid measures are available, where to find water, where to dump waste, and so on.}
\end{enumerate}

\section{Knowing Thy Syllabus}
This guide provides advice and instructions for performing a wide range of experiments in just about every topic studied in secondary school.  However, your country’s national curriculum will be more specific in telling you what areas to focus on, what skills are expected of students, and perhaps even specific experiments that students will need to perform or understand for their exams.

Once you have taught for a few weeks and have a clearer idea of your students’ learning styles and ability levels, sit down with another science teacher and try to answer as many of the following questions as you can:
\begin{enumerate}
\item{How is each science subject area (physics, chemistry, biology, earth science) divided up among the grade levels?}
\item{What are the major topic areas for each subject and each grade level?}
\item{Before you ask the question “which topics could be done in the lab?” stop, take a breath, and tell yourself this:  All of them can be done.  And quite easily, at that.  You’re not looking at the syllabus to pick and choose (though you will at the beginning); you’re getting a glimpse of the diverse greatness that will be your science classroom.}
\item{Identify the most challenging topics.  Many teachers have their favorite topics and those that they shy away from; these tend to be things like qualitative analysis (chemistry), electromagnetism (physics), dissections (biology), and many more.  Often, when trying to get interest from staff, it is effective to show that these topics can be demonstrated simply.}
\item{Get a feel for the exams.  Most countries have a single national exam that all students take at the end of secondary school, and there will be trends in the types of experiments they expect the students to know. You don’t want to teach to the test always, but the education authorities chose those experiments for a good reason and they probably incorporate many of the skills that you will want your students to be comfortable with.}
\end{enumerate}

\section{Knowing Thy Self}
It is important to go into this project at your own pace.  Many PCVs were not trained in school to experiments on a regular basis with basic materials.  If you feel overwhelmed by the sheer scale of the task before you, don’t worry: you’re not alone.  Also, we’re going to take it nice a slow, one thing at a time.

After you’ve familiarized yourself with the school, students, syllabus, and community, and before you launch head-first into the village trash pits looking for dead batteries, take a step back and decide how you want to go about it.  What is your personal experience with and interest in hands-on science?  What would your teaching be like without it, and what would the students gain from it?  What are your skills when it comes to making solutions and gadgets?  And how dazzled will your students be the first time they blow up a hydrogen balloon and scare the dickens out of the teacher next door?

Page through some of the activities in this guide to see the types of materials and activities we recommend.  Note one or two that you think will be really simple and fun: these will be your first.

\section{Getting the Goods}
Now for the fun part!  While it may be tempting to go out on the town, buckets and cash in hand, and bring home the mother lode, it’s important to remember to start small.  Pick a few demonstrations you’d like to do with your class in the next week and locate the materials you’ll need.  The sections about \nameref{cha:sourcesofchemicals} and \nameref{labequip} gives detailed descriptions of where to get these materials or functional alternatives for a low price.

Your first resource is your immediate surroundings; with a certain demonstration in mind, take a leisurely stroll around your house, school, and to the nearby shops, keeping an eye out for anything that might serve your purpose.  You may be pleasantly surprised how much useful stuff is just lying around: plastic water bottles are some of the most useful items you will ever pick up off the ground, as are dead radio (D-cell) batteries, basic cooking ingredients, and broken electronics.  Gather these and anything else that looks remotely interesting.  People might wonder what you’re up to; a short explanation may cause them to wonder at the strange behavior of Americans, or may get them excited to help.  Students and community members, on hearing the call for more junk for the new school lab, have been known to deliver homemade devices, old car stereos, and even pig eyes for dissection.

Once you have tried your hand at scavenging the immediate environs goods, and you have tested the materials in action either at home or with students, you may want to try your hand at the big time.  Make a list of materials you need but can’t find in your community and try looking for them the next time you visit town.  Chances are they won’t be free, but a glance at the < > section will show you that they can probably be found for a very low price.  These materials tend to include chemicals, a wider array of broken electronics (a gold mine for the physics syllabus), glass, etc.  Carry or purchase a bucket each time you go to town as you never know what you will find there.

\section{Setting up Shop}
One of the most important drivers for continued exploration of hands-on science experiments is a designated space.  An open work bench with all your new toys arranged and labeled is much more encouraging than a pile of rubbish in the corner next to all those papers you still haven’t graded.  Take the time (and money) to set up a space conducive to your working style, make yourself a cup of tea, and start playing.

Once you have your space and a couple choice demos packed up and ready to go, you’re well on your way to making the lab.  Begin with the simple demonstrations for the topics you are teaching and commit to continuing that trend for every topic you cover.  Establish a routine of bringing practical science into the classroom on a regular basis, no matter how involved, potentially every day.  The benefit of these simple activities is that students will see science being done, not in a special setting with expensive equipment, but in their very own classroom using materials they see every day.  This can be a powerful way to help your students invest in what they are learning, and will set you up to begin putting the demonstrations in their hands.

\section{Up-Scaling}

Being comfortable with regular science demonstrations as a focus for your teaching is one thing; creating a laboratory setting is quite another.  It will require support from the school, organization, and more time to acquire materials in bulk.  A good way to begin is to turn your classroom into a laboratory once in a while, perhaps as an after-school session, and run an experiment with smaller groups of students.  Once this becomes comfortable in terms of classroom management, safety, and overall success of the activity, you may want to look into creating permanent laboratory space for other teachers and students to use.

\subsection{More Goods}
Up to now you have probably established quite a collection of widgets and junk to satisfy your students’ appetite for interesting demonstrations.  Take stock of what you have and what you need more of if you are to do experiments with entire classes on a regular basis.  This may seem like way too much at first glance, but because you are using simple materials most things can have multiple uses!  Also, many of the chemicals you will use can be found cheaply in concentrated form as per the < > section and diluted (safely) into a supply that your old high school would envy.  Some common materials you will need to keep on a classroom-scale are:
\begin{itemize}
\item{Plastic water bottles (variable size)}
\item{Dead D-cells}
\item{Disposable syringes}
\item{Various salts, acids, bases}
\item{Clean, soft water}
\item{Rulers, stopwatches}
\item{Receptacles (beakers, jars, bottles, buckets, etc.)}
\item{Rags}
\item{Heat sources}
\item{Wires}
\end{itemize}

\subsection{Bigger Shop}
Most schools have a designated lab space, or at least want one, but in either case you will need to establish a room in the school as the laboratory.  Inside you will need work benches with stools, secure storage space for all the toys and chemicals, and plenty of daylight.  Offer to purchase padlocks for the door and cabinets, and suggest that the school install wire mesh around the windows if theft is a potential problem (though, hopefully, no single item in your lab costs more than the lock itself).  Establish a schedule with the other teachers and work with them to arrange the space effectively.

\subsection{Safety First}
In any laboratory safety is of the utmost importance, and the guidelines in the \nameref{labsafety} section should be read thoroughly before beginning any lab activities with the students.  Additionally, it may be helpful to consider that most of your students, and perhaps even the teachers, are not accustomed to working in a laboratory setting.  For this reason they may be more receptive to laboratory rules and regulations, being fully aware of their inexperience with dangerous chemicals.  Before entering the lab each day, remind your students of the potential risk involved in the day’s experiment as well as basic lab rules in general.  Once they are accustomed to this, have them repeat the rules themselves before entering.  Inside the lab itself, post the safety rules in an obvious location, as well as instructions for what to do in the case of a spill or other dangerous situation.  Provide a hand-washing station near the door, with soap and a clean towel.  Have dilute acid and base solutions on hand in case of a spill.  Remember to label and organize everything in its proper place, and instruct students to do the same and value the neatness of the laboratory.

\section{Now What?}
No laboratory is perfect, nor is it ever truly complete.  If your school has taken ownership of this laboratory (hopefully it is no longer in your control) and plans to use it and expand it, help to lead the project in a sustainable direction.  While hands-on science is a nice thing to have, many schools will still value fancy, expensive equipment over the vulgar junk you dragged in.  There is certainly value in having these luxuries if the school can afford it, but help to foster the understanding that the benefit of local materials is their renewable nature; experiments can be performed often without fear of breaking any apparatus or running out of any chemical.

Meanwhile, however, you have an opportunity to see countless experiments performed by countless students, providing a unique insight into what is working and what is not.  Refine the laboratory practices; rewrite any experiments that need editing or updating; continue the search for more resources and practical applications.  If possible, expand the reach of your hands-on-science empire to other schools, offering to host other students or teachers for science workshops, create science clubs and inter-school science competitions.  Chances are that, if your school was struggling to bring hands-on science into the classroom, other schools are as well.

The longevity of this laboratory lies in the teachers and students.  If students show an interest, help train them to be lab technicians so that they can run simple experiments in your absence, help keep inventory, and design new demonstrations for their younger peers.  These are the students who will go on to start laboratories of their own.  The teachers, also, will have much to contribute to the cause and will ultimately provide the most refinement to the operation of the lab.
