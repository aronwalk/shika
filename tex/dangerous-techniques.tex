\chapter{Dangerous Techniques}

Some common laboratory techniques are actually quite dangerous. 
Identify practices in your school 
that seem likely to cause harm and devise safer alternatives. 
Below are two examples of techniques often performed in the laboratory 
that can easily bring harm 
and alternative methods to do the same thing more safely.

\section{Mouth pipetting}

Many schools use pipettes for titrations. 
Many students use their mouths to fill these pipettes. 
We strongly discourage this practice.

The solutions used in ordinary acid-base titrations 
are not particularly dangerous. 
A little 0.1M NaOH in the mouth 
does not merit a trip to the hospital. 
Nevertheless, 
there are two pressing safety issues. 
First, 
there are often other solutions present on the same benches – 
the qualitative analysis test reagents for example – 
that can kill if consumed. 
It seems like it would be a rare event 
for a student to mix up the bottles, 
but in the panic of the exam anything is possible.

The second safety issue applies to the best students, 
those that continue on to more advanced levels. 
High level secondary and university students 
must measure volumes of the size fit for pipettes 
for chemicals that under no circumstances should be mouth pipetted. 
If a student is trained in mouth pipetting, 
she will continue with this habit in advanced level, 
especially in a moment of frustration 
when a pipette filling bulb seems defective, 
or if the school has not taught her how to use them, 
or if they are not supplied. 
Students have died in many countries from mouth pipetting toxins. 
Pipetting is another instance 
when doing something without the rubber is a bad plan.

Fortunately, 
there is no reason to ever use a pipette in secondary school, 
even if rubber-filling bulbs are present. 
Disposable plastic syringes are in every way superior 
to pipettes for the needs of students. 
First, 
they have no risk of chemical ingestion. 
Second, 
they are more accurate – 
plastic is much easier to make standard size than glass; 
the pipettes available general vary from their true volume, 
but all the syringes of the same model 
and maker are exactly the same volume. 
A third advantage is that plastic syringes are easier to use. 
Fourth, 
they are faster to use. 
Fifth, 
they are much more durable and, 
sixth, 
when they do break they make no dangerous shards. 
Last, 
and truly least, 
they are much less expensive, 
by about an order of magnitude. 
Schools all over are already substituting plastic syringes for glass pipettes.
For information on how to use these plastic syringes, 
please see How to Use a Plastic Syringe.

\section{Shaking separatory funnels}

Separatory funnels are useful for separating immiscible liquids. 
They are also made of glass, 
very smooth, 
and prone to slipping out of students' hands. 
The liquids often used in these funnels 
can be quite harmful and no one wants them 
splashed along with glass shards on the floor.

Much better is to add the mixture to a plastic water bottle, 
cap it tightly, 
and shake. 
After shaking, 
transfer the contents of the bottle into a narrow beaker. 
Either layer can be efficiently removed with a plastic syringe.

There are some cases where a separatory funnel remains essential. 
For secondary school, 
however, 
simply design experiments that use other equipment - 
and less harmful chemicals.

\section{Looking down into test tubes}
May blind.
